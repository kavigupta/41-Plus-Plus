\documentclass{article}
\usepackage{verbatim}
\usepackage{listings}

\newcommand{\code}[1]{\texttt{#1}}
\newcommand{\codeblock}[1]{\begin{quote}\code{#1}\end{quote}}

\begin{document}
\section{Values}
\subsection{Literals}
Literals are expressions that are hard-coded into the code. They take one of four forms.
\subsection{Numeric Literals}
These must start with a digit, a plus sign, a minus sign, or a period.
\subsection{Boolean Literals}
Either \code{true} or \code{false}.
\subsection{Character and String Literals}
These must start and end with a single quote \code{'}. What is in between is interpreted as a string. To use an actual single quote mark, use \code{\textbackslash'}. Standard escapes can also be used. Determining the type of a string falls into three cases.
\begin{enumerate}
\item \code{''}: This is automatically a string literal representing an empty string.
\item A single character: depending on the context, this is interpreted as a string or character.
\item Multiple characters: always a string.
\end{enumerate}
\subsection{Examples}
\begin{itemize}
\item \code{2}, \code{-56543234565}, \code{41}, \code{-.02345654321}, \code{12.}: Numeric literals.
\item \code{'"'}, \code{'1'}, \code{'\r'}, \code{'\n'}, \code{'\t'}, \code{'\0123'}: Character literals
\item \code{''}, \code{'\textbackslash'\textbackslash''}, \code{'41++'}: String literals.
\end{itemize}
\section{Statements}
There are a limited number of valid statement forms. All start with a capital letter and end with a period.
\subsection{Definition}
\subsubsection{Declaration}
A minimal declaration simply provides a variable with a name and associates it with a type.

\codeblock{Define a[n] <type> called <name>.}

\noindent This is equivalent to the Java \code{<type> <name>;} 

\subsubsection{Field Initialization}
A variable can also have its fields initialized, including the field \code{value}, which represents the value of the entire structure.

\codeblock{Define a[n] <type> called <name> with a[n] <field1> of <value1>, a[n] <field2> of <value2>, and a[n] <field3> of <value3>.}

\noindent Commas and \code{and} are all technically unnecessary, but included to insure readability. Similarly, \code{a} and \code{an} are equivalent but both are included to avoid statements like \code{Define a integer called x.}
\subsubsection{Examples}

\codeblock{Define an integer called x.}

\codeblock{Define a string called name with a value of ``41++''.}

\codeblock{Define a matrix called M with a width of 3 and a height of 2. Define a matrix called M2 with a value of M.}

\end{document}